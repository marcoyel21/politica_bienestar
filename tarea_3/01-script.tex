% Options for packages loaded elsewhere
\PassOptionsToPackage{unicode}{hyperref}
\PassOptionsToPackage{hyphens}{url}
%
\documentclass[
  11pt,
]{article}
\usepackage{amsmath,amssymb}
\usepackage{lmodern}
\usepackage{ifxetex,ifluatex}
\ifnum 0\ifxetex 1\fi\ifluatex 1\fi=0 % if pdftex
  \usepackage[T1]{fontenc}
  \usepackage[utf8]{inputenc}
  \usepackage{textcomp} % provide euro and other symbols
\else % if luatex or xetex
  \usepackage{unicode-math}
  \defaultfontfeatures{Scale=MatchLowercase}
  \defaultfontfeatures[\rmfamily]{Ligatures=TeX,Scale=1}
\fi
% Use upquote if available, for straight quotes in verbatim environments
\IfFileExists{upquote.sty}{\usepackage{upquote}}{}
\IfFileExists{microtype.sty}{% use microtype if available
  \usepackage[]{microtype}
  \UseMicrotypeSet[protrusion]{basicmath} % disable protrusion for tt fonts
}{}
\makeatletter
\@ifundefined{KOMAClassName}{% if non-KOMA class
  \IfFileExists{parskip.sty}{%
    \usepackage{parskip}
  }{% else
    \setlength{\parindent}{0pt}
    \setlength{\parskip}{6pt plus 2pt minus 1pt}}
}{% if KOMA class
  \KOMAoptions{parskip=half}}
\makeatother
\usepackage{xcolor}
\IfFileExists{xurl.sty}{\usepackage{xurl}}{} % add URL line breaks if available
\IfFileExists{bookmark.sty}{\usepackage{bookmark}}{\usepackage{hyperref}}
\hypersetup{
  hidelinks,
  pdfcreator={LaTeX via pandoc}}
\urlstyle{same} % disable monospaced font for URLs
\usepackage[margin=1in]{geometry}
\usepackage{longtable,booktabs,array}
\usepackage{calc} % for calculating minipage widths
% Correct order of tables after \paragraph or \subparagraph
\usepackage{etoolbox}
\makeatletter
\patchcmd\longtable{\par}{\if@noskipsec\mbox{}\fi\par}{}{}
\makeatother
% Allow footnotes in longtable head/foot
\IfFileExists{footnotehyper.sty}{\usepackage{footnotehyper}}{\usepackage{footnote}}
\makesavenoteenv{longtable}
\usepackage{graphicx}
\makeatletter
\def\maxwidth{\ifdim\Gin@nat@width>\linewidth\linewidth\else\Gin@nat@width\fi}
\def\maxheight{\ifdim\Gin@nat@height>\textheight\textheight\else\Gin@nat@height\fi}
\makeatother
% Scale images if necessary, so that they will not overflow the page
% margins by default, and it is still possible to overwrite the defaults
% using explicit options in \includegraphics[width, height, ...]{}
\setkeys{Gin}{width=\maxwidth,height=\maxheight,keepaspectratio}
% Set default figure placement to htbp
\makeatletter
\def\fps@figure{htbp}
\makeatother
\setlength{\emergencystretch}{3em} % prevent overfull lines
\providecommand{\tightlist}{%
  \setlength{\itemsep}{0pt}\setlength{\parskip}{0pt}}
\setcounter{secnumdepth}{-\maxdimen} % remove section numbering
\usepackage{pdflscape} \usepackage{geometry} \usepackage{graphicx} \tolerance=1 \hyphenpenalty=10000 \hbadness=10000 \linespread{1.3} \usepackage[justification=centering, font=bf, labelsep=period, skip=5pt]{caption} \usepackage{titling} \usepackage[spanish]{babel} \usepackage{fancyhdr} \pagestyle{fancy} \fancyhead[L]{Maestría en Economía Aplicada} \fancyhead[R]{ITAM} \usepackage{float}
\usepackage{booktabs}
\usepackage{longtable}
\usepackage{array}
\usepackage{multirow}
\usepackage{wrapfig}
\usepackage{float}
\usepackage{colortbl}
\usepackage{pdflscape}
\usepackage{tabu}
\usepackage{threeparttable}
\usepackage{threeparttablex}
\usepackage[normalem]{ulem}
\usepackage{makecell}
\usepackage{xcolor}
\ifluatex
  \usepackage{selnolig}  % disable illegal ligatures
\fi
\newlength{\cslhangindent}
\setlength{\cslhangindent}{1.5em}
\newlength{\csllabelwidth}
\setlength{\csllabelwidth}{3em}
\newenvironment{CSLReferences}[2] % #1 hanging-ident, #2 entry spacing
 {% don't indent paragraphs
  \setlength{\parindent}{0pt}
  % turn on hanging indent if param 1 is 1
  \ifodd #1 \everypar{\setlength{\hangindent}{\cslhangindent}}\ignorespaces\fi
  % set entry spacing
  \ifnum #2 > 0
  \setlength{\parskip}{#2\baselineskip}
  \fi
 }%
 {}
\usepackage{calc}
\newcommand{\CSLBlock}[1]{#1\hfill\break}
\newcommand{\CSLLeftMargin}[1]{\parbox[t]{\csllabelwidth}{#1}}
\newcommand{\CSLRightInline}[1]{\parbox[t]{\linewidth - \csllabelwidth}{#1}\break}
\newcommand{\CSLIndent}[1]{\hspace{\cslhangindent}#1}

\author{}
\date{\vspace{-2.5em}}

\begin{document}

\begin{titlepage}
\begin{center}

\textsc{\Large Instituto Tecnológico Autónomo de México}\\[2em]

\textbf{\LARGE Bienestar y Política Social}\\[2em]


\textsc{\LARGE }\\[1em]


\textsc{\large Tarea 3 }\\[1em]

\textsc{\LARGE }\\[1em]

\textsc{\large }\\[1em]
\textsc{\LARGE }\\[1em]
\textsc{\LARGE }\\[1em]

\textsc{\large }\\[1em]
\textsc{\LARGE }\\[1em]

\textsc{\LARGE Dra. Araceli Ortega Díaz}\\[1em]

\textsc{\LARGE }\\[1em]
\textsc{\LARGE }\\[1em]

\textsc{\LARGE Marco Antonio Ramos Juárez}\\[1em]

\textsc{\large 142244}\\[1em]

\textsc{\LARGE Mayra Samantha Cervantes Bravo}\\[1em]

\textsc{\large 141371}\\[1em]

\textsc{\LARGE Cynthia Raquel Valdivia Tirado }\\[1em]

\textsc{\large 81358}\\[1em]

\end{center}

\vspace*{\fill}
\textsc{Ciudad de México \hspace*{\fill} 2021}

\end{titlepage}

\newpage

\tableofcontents

\newpage

\hypertarget{introducciuxf3n}{%
\section{Introducción}\label{introducciuxf3n}}

En esta tarea analizaremos la relación entre desigualdad y crecimiento
económico desde un punto de vista teórico pero también econométrico.
Asimismo, exploraremos el efecto que tiene el nivel de ingreso de los
países en la desigualdad y el crecimiento. En este sentido, en la
primera parte expondremos un marco teórico muy breve acompañado de un
modelo econométrico donde la variable independiente es el coeficiente de
Gini y la variable dependiente el crecimiento del PIB per capita.
Posteriormente, se realizarán pruebas de hipótesis para determinar si se
debe utilizar un modelo \emph{pooled} o de efectos fijos, así como si se
debe controlar por el tiempo. En la segunda parte, extenderemos el
análisis para incluir el nivel de ingreso (alto, mediano o bajo ) en los
modelos, con las mismas pruebas de hipótesis. Finalmente, se hará una
breve conclusión de los hallazgos.

\hypertarget{relaciuxf3n-entre-crecimiento-desigualdad-y-nivel-de-ingreso}{%
\section{Relación entre crecimiento, desigualdad y nivel de
ingreso}\label{relaciuxf3n-entre-crecimiento-desigualdad-y-nivel-de-ingreso}}

\hypertarget{argumentos-teuxf3ricos}{%
\subsection{Argumentos teóricos}\label{argumentos-teuxf3ricos}}

La relación entre crecimiento y desigualdad es un tema muy contencioso
en la economía, pues existe evidencia teórica y empírica que encuentra
que la relación es tanto negativa como positiva.\footnote{Heather
  Boushey y Carter C Price, {``{How Are Economic Inequality and Growth
  Connected? A review of recent research}''}, \emph{Washington Center
  for Equitable Growth}, núm. October 2014 (2014): 1--25,
  \href{https://www.equitablegrowth.org}{www.equitablegrowth.org}.}

Desde un punto de vista teórico, el economista Robert Barro sostiene que
existe una curva de Kuznets que relaciona al crecimiento con la
desigualdad: mientras que al principio del desarrollo económico la
desigualdad tiende a crecer, esta eventualmente decrece con la
maduración de la economía. Sin embargo, como reconoce el premio Nobel,
la relación entre ambas variables es modesta.\footnote{Robert J. Barro,
  {``{Inequality and growth in a panel of countries}''}, \emph{Journal
  of Economic Growth}, 2000,
  \url{https://doi.org/10.1023/A:1009850119329}.}

Asimismo, Barro argumenta que el crecimiento podría estar determinado
por el nivel de desarrollo de un país. En los países pobres, la
desigualdad inhibe el crecimiento, mientras que en los países ricos, la
desigualdad promueve el crecimiento. Por esta razón consideramos que
para evaluar el efecto de la desigualdad en el crecimiento, forzozamente
necesitamos evaluar en conjunto con el nivel de ingreso.

\hypertarget{argumentos-economuxe9tricos}{%
\subsection{Argumentos
econométricos}\label{argumentos-economuxe9tricos}}

En primer lugar para cuantificar la relación entre crecimiento y la
desigualdad, realizamos una serie de modelos sencillos donde vemos el
efecto que tiene el coeficiente de Gini como única variable
independiente frente al crecimiento del PIB per Capita. Para esta tarea
descargamos los datos de las \emph{World Penn Tables} y los combinamos
con una base de datos limpiada con los coeficientes de Gini del Banco
Mundial. Debido a la disponibilidad de datos, se decidió usar solo
información de 98 países desde 1990 hasta 2015. De esta manera obtenemos
una base de datos perfectamente balanceada.

\hypertarget{modelos-univariados}{%
\section{Modelos univariados}\label{modelos-univariados}}

\begin{verbatim}
                                              Dependent variable:                                     
         ---------------------------------------------------------------------------------------------
                                                   growth_pc                                          
                Pooled        Efectos Fijos (individuales) Efectos Fijos (two-ways) Efectos Aleatorios
                 (1)                      (2)                        (3)                   (4)        
\end{verbatim}

\begin{longtable}[]{@{}
  >{\raggedright\arraybackslash}p{(\columnwidth - 0\tabcolsep) * \real{1.00}}@{}}
\toprule
\endhead
gini -0.0001 -0.003*** -0.003*** -0.0003 (0.0002) (0.001) (0.001)
(0.0002) \\ \addlinespace
Constant 0.031*** 0.039*** (0.007) (0.009) \\ \addlinespace
\bottomrule
\end{longtable}

Observations 2,416 2,416 2,416 2,416\\
R2 0.0001 0.015 0.013 0.001\\
Adjusted R2 -0.0004 -0.026 -0.039 0.0003\\
F Statistic 0.142 (df = 1; 2414) 36.304*** (df = 1; 2317) 31.289*** (df
= 1; 2293) 1.577\\
==========================================================================================================
Note: \emph{p\textless0.1; \textbf{p\textless0.05; }}p\textless0.01
Podemos observar que el coeficiente de la variable Gini tiene signo
negativo en los cuatro modelos, aunque en el modelo \emph{pooled} y de
efectos aleatorios no tiene significancia estadística, mientras que en
los modelos within con efectos individuales y two-ways sí.

\hypertarget{pruebas-de-hipuxf3tesis}{%
\subsection{Pruebas de hipótesis}\label{pruebas-de-hipuxf3tesis}}

\begin{table}[H]
\centering
\begin{tabular}{ll}
\toprule
Pruebas & P.value\\
\midrule
Prueba F: Pooled vs EF individual & 0.0000\\
Hausman: EF vs EA & 0.0000\\
Breusch-Pagan: EF individuales vs two-way & 0.0000\\
\bottomrule
\end{tabular}
\end{table}

Dado que el modelo MCO o \emph{pooled} podría generar estimadores
sesgados si no se cumple con los supuestos de linealidad, exogeneidad,
homocedasticidad, y no multicolinealidad, se deben realizar pruebas para
verificar que el efecto individual \(u_{i}\) no existe, o es igual a
cero. Esto se realiza con una prueba F, donde \emph{Ho} es que los
efectos individuales, o errores \(u_{i}\) son iguales a cero. Dado que
se rechaza a un nivel de significancia de .0000, podemos descartar el
modelo \emph{pooled}.

En segunda instancia, se realiza una prueba para determinar si el modelo
de efectos fijos o de efectos aleatorios es preferible. Realizamos una
prueba de Hausman donde \emph{Ho} es que los errores \(u_{i}\) no están
correlacionados con los regresores. Este sería el caso de un modelo con
efectos aleatorios. Al rechazar \emph{Ho} con un nivel de significancia
de .0000 , se implica que los errores están correlacionados a los
regresores, o que hay efectos fijos en el panel.

Por último, es necesario concluir si se deben incluir, además efectos
fijos individuales, efectos fijos en el tiempo (\emph{two-ways}). Esto
se realiza con una prueba de multiplicadores lagrangianos Breusch-Pagan,
en la que se determina si se debe incluir el término \(lambda_{t}\), o
efectos tiempo-específicos. \emph{Ho} implica que hay variables con
efectos específicos individuales, pero no con efectos específicos en el
tiempo. Al rechazar \emph{Ho} a un nivel del 0.000, concluimos que hay
efectos de tiempo significativos.

De estas pruebas concluimos que el modelo de efectos fijos es más
adecuado que el de \emph{pooled} y efectos aleatorios, a partir de las
pruebas F y de Hausman, respectivamente; y que del modelo de efectos
fijos individual al two-way el segundo es más adecuado, ya que se probó
con el test de Breusch-Pagan que en la ecuación de efectos fijos:

\[\y_{it}=x_{it}\beta+c_i+u_{it} +\lambda_{t}\],

\$\lambda\_\{t\} \neq 0.

\hypertarget{modelos-con-nivel-de-ingreso}{%
\section{Modelos con nivel de
ingreso}\label{modelos-con-nivel-de-ingreso}}

En segundo lugar, realizaremos el mismo ejercicio pero ahora controlando
por la interacción del nivel de ingreso de los países (alto, medio y
bajo) con el coeficiente de Gini. Esta interacción captará si la
desigualdad afecta a las colas de la distribución del ingreso de
diferentes formas.

\hypertarget{defensa-economuxe9trica}{%
\subsection{Defensa Econométrica}\label{defensa-economuxe9trica}}

\begin{table}[!htbp] \centering 
  \caption{Modelos multivariados} 
  \label{} 
\small 
\begin{tabular}{@{\extracolsep{3pt}}lcccc} 
\\[-1.8ex]\hline 
\hline \\[-1.8ex] 
 & \multicolumn{4}{c}{\textit{Dependent variable:}} \\ 
\cline{2-5} 
\\[-1.8ex] & \multicolumn{4}{c}{growth\_pc} \\ 
 & Pooled & Efectos Fijos (individuales) & Efectos fijos (twoways) & Efectos aleatorios \\ 
\\[-1.8ex] & (1) & (2) & (3) & (4)\\ 
\hline \\[-1.8ex] 
 gini & $-$0.0001 & $-$0.005$^{***}$ & $-$0.004$^{***}$ & $-$0.0002 \\ 
  & (0.0002) & (0.001) & (0.001) & (0.0002) \\ 
  & & & & \\ 
 I(gini \textasteriskcentered  inc\_high) & $-$0.0001 & 0.006$^{***}$ & 0.005$^{***}$ & $-$0.0002 \\ 
  & (0.0001) & (0.002) & (0.002) & (0.0001) \\ 
  & & & & \\ 
 I(gini \textasteriskcentered  inc\_low) & $-$0.001$^{***}$ & 0.004$^{***}$ & 0.005$^{***}$ & $-$0.001$^{***}$ \\ 
  & (0.0001) & (0.002) & (0.002) & (0.0002) \\ 
  & & & & \\ 
 Constant & 0.034$^{***}$ &  &  & 0.041$^{***}$ \\ 
  & (0.008) &  &  & (0.009) \\ 
  & & & & \\ 
\hline \\[-1.8ex] 
Observations & 2,416 & 2,416 & 2,416 & 2,416 \\ 
R$^{2}$ & 0.008 & 0.023 & 0.021 & 0.006 \\ 
Adjusted R$^{2}$ & 0.007 & $-$0.019 & $-$0.032 & 0.005 \\ 
F Statistic & 6.886$^{***}$ (df = 3; 2412) & 18.094$^{***}$ (df = 3; 2315) & 16.020$^{***}$ (df = 3; 2291) & 14.568$^{***}$ \\ 
\hline 
\hline \\[-1.8ex] 
\textit{Note:}  & \multicolumn{4}{r}{$^{*}$p$<$0.1; $^{**}$p$<$0.05; $^{***}$p$<$0.01} \\ 
\end{tabular} 
\end{table}

En estos modelos podemos notar que el efecto de la desigualdad es
negativo para los países que no pertenecen ni al grupo de ingreso alto
ni al de bajo, es decir, a los de en medio. Por el contrario, el efecto
de la desigualdad en los países de ingreso alto es positivo respecto a
los países de ingreso medio, con un coeficiente mayor a cero. Sin
embargo, este efecto en los países de ingresos bajos, aunque mayores a
los de ingresos medios, no son positivos.

Esto corrobora y contrasta la teoría de Barro. Por un lado, nuestros
hallazgos son congruentes con su hipótesis de que la desigualdad
incentiva al crecimiento económico en países ricos. Sin embargo, en
países pobres no vemos una relación tan clara contrario a la hipótesis
de que en estos países el efecto sería negativo: parece que la
desigualdad es perjudicial para el crecimiento económico,
particularmente en los países de ingresos medios.

Esto podría deberse a que tal vez Barro se refería a los países de
ingreso medio mas que a los de ingreso bajo (pues estos ni siquiera
habrían despegado económicamente o a que tal vez en lugar de una curva
de Kuznets la forma funcional sea una curva en forma de S).

\hypertarget{pruebas-de-hipuxf3tesis-1}{%
\subsection{Pruebas de hipótesis}\label{pruebas-de-hipuxf3tesis-1}}

\begin{table}[H]
\centering
\begin{tabular}{ll}
\toprule
Pruebas & P.value\\
\midrule
Prueba F: Pooled vs EF individual & 0.0000\\
Hausman: EF vs EA & 0.0000\\
Breusch-Pagan: EF individuales vs two-way & 0.0000\\
\bottomrule
\end{tabular}
\end{table}

En esta sección, se realizó de nuevo la prueba del modelo de efectos
fijos contra \emph{pooled}, y de nuevo se rechazó Ho a favor del
primero. Adicionalmente, se hizo una prueba de multiplicador lagrangiano
Breusch-Pagan para efectos de tiempo fijos, donde Ho es que
\(\lambda_{t} \= 0\). Al rechazar, concluimos que es necesario controlar
por tiempo en el modelo de efectos fijos, optando por el modelo
two-ways.

\hypertarget{modelo-final}{%
\section{Modelo final}\label{modelo-final}}

\begin{table}[!htbp] \centering 
  \caption{Modelo multivariado} 
  \label{} 
\small 
\begin{tabular}{@{\extracolsep{3pt}}lc} 
\\[-1.8ex]\hline 
\hline \\[-1.8ex] 
 & \multicolumn{1}{c}{\textit{Dependent variable:}} \\ 
\cline{2-2} 
\\[-1.8ex] & growth\_pc \\ 
 & Efectos Fijos (two-way) \\ 
\hline \\[-1.8ex] 
 gini & $-$0.004$^{***}$ \\ 
  & (0.001) \\ 
  & \\ 
 I(gini \textasteriskcentered  inc\_high) & 0.005$^{***}$ \\ 
  & (0.002) \\ 
  & \\ 
 I(gini \textasteriskcentered  inc\_low) & 0.005$^{***}$ \\ 
  & (0.002) \\ 
  & \\ 
\hline \\[-1.8ex] 
Observations & 2,416 \\ 
R$^{2}$ & 0.021 \\ 
Adjusted R$^{2}$ & $-$0.032 \\ 
F Statistic & 16.020$^{***}$ (df = 3; 2291) \\ 
\hline 
\hline \\[-1.8ex] 
\textit{Note:}  & \multicolumn{1}{r}{$^{*}$p$<$0.1; $^{**}$p$<$0.05; $^{***}$p$<$0.01} \\ 
\end{tabular} 
\end{table}

Las pruebas de hipótesis mostradas muestran la bondad de ajuste del
modelo final, el within two-way, que incluye a la variable de tiempo
(\(\lambda_{t}\)) en su estimación.

Esta es la forma más eficiente de capturar los efectos fijos del panel,
comparada con el modelo de \emph{least squares dummy variables} o el
modelo within individual incluyendo variables de tiempo,\footnote{\textbf{Baltagi2013?}}
ya que estos dan errores estándar incorrectos debido al número mayor de
grados de libertad, así como que la \(R^2\) es correcta.\footnote{\textbf{Park2011?}}
La inclusión de \(\lambda_{t}\) podría señalar eventos que sucedieron en
el tiempo del panel que afectaron al crecimiento económico entre los
países, como tratados de libre comercio o crisis económicas de los
países.

Los resultados señalan que el coeficiente de Gini tiene una relación
negativa y significativa sobre el crecimiento económico per cápita, y
como mencionamos anteriormente, que en los países pobres la desigualdad
afecta al crecimiento económico, mientras que en los ricos existe una
relación positiva entre ellas.

Una teoría acorde a la de nuestros resultados es la \emph{teoría de la
acumulación de capital humano} que liga a la desigualda y al crecimiento
económico: la desigualdad de ingresos deprime el desarrollo de
habilidades de individuos cuyos padres tienen niveles bajos de
educación, tanto en cantidad y calidad educativa. Mientras tanto, los
resultados educativos de individuos más ricos no están afectados por la
desigualdad.\footnote{\textbf{Cingano2014?}}

\hypertarget{conclusiones}{%
\section{Conclusiones}\label{conclusiones}}

En conclusión, con los datos a nuestra disposición y bajo la
especificación elegida, la relación entre el crecimiento y la
desigualdad fue la misma en los diferente modelos con la especificación
más simple: el coeficiente de Gini tiene una relación negativa con el
crecimiento económico per cápita. Sin embargo, perdía fuerza estadística
en los modelos \emph{pooled} y de efectos aleatorios.

Sin embargo, al introducir la variable de interacción entre desigualdad
y nivel de ingreso, sucedieron dos cosas: por un lado, pudimos discernir
el efecto de ella (la desigualdad) a través de la distribución del
ingreso, observando que no afecta de forma igual a los diferentes
países; por el otro, que la relación era ambigua y con poco poder
explicativo, ya que, mientras que en los modelos within los países de
ingresos altos y bajos tenían signo positivo del coeficiente respecto a
los de ingresos medios, sucedía lo contrario en el modelo \emph{pooled}
y de efectos aleatorios.

Para escoger un modelo ``más adecuado'' y ajustado a los datos,
realizamos múltiples pruebas para evaluar la existencia de efectos fijos
entre entidades, en el tiempo, y si existía correlación entre los
errores de los países y las variables regresoras. Los resultados
concluyeron que el modelo con mejor bondad de ajuste (más eficiente) era
el de efectos fijos \emph{two-ways} .

Una conclusión a la que podemos llegar a partir de estos resultados es
que las políticas para reducir la desigualdad son clave, sobre todo en
países pobres, no solo para mejorar el aspecto social del país, si no
para asegurar un crecimiento económico a largo plazo. Esto se podría
lograr a través de impuestos y transferencias, lo cual no necesariamente
dañaría al crecimiento.\footnote{\textbf{Cingano2014?}}

Finalmente algo muy importante es que los resultados dependen totalmente
de la especificación del modelo de regresión así como de las variables
que usemos como control. Esto no es un hallazgo nuevo pues diversos
autores sostienen que tanto la especificación como la calidad de los
datos han hecho que las conclusiones en este tema sean
ambivalentes.\footnote{Boushey y Price, {``{How Are Economic Inequality
  and Growth Connected? A review of recent research}''}.} Esto sugiere
que en este tema la relación no es tan inmediata y que se requiere de
más trabajo análitico y teórico, y posiblemente hasta en la calidad de
los datos.

\newpage

\hypertarget{referencias}{%
\section*{Referencias}\label{referencias}}
\addcontentsline{toc}{section}{Referencias}

\hypertarget{refs}{}
\begin{CSLReferences}{1}{0}
\leavevmode\hypertarget{ref-Barro2000}{}%
Barro, Robert J. {``{Inequality and growth in a panel of countries}''}.
\emph{Journal of Economic Growth}, 2000.
\url{https://doi.org/10.1023/A:1009850119329}.

\leavevmode\hypertarget{ref-Boushey2014}{}%
Boushey, Heather, y Carter C Price. {``{How Are Economic Inequality and
Growth Connected? A review of recent research}''}. \emph{Washington
Center for Equitable Growth}, núm. October 2014 (2014): 1--25.
\href{https://www.equitablegrowth.org}{www.equitablegrowth.org}.

\end{CSLReferences}

\end{document}
